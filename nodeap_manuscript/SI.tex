\documentclass[a4paper,12pt]{article}
\usepackage{amsmath}
\usepackage{amssymb} % For symbols like \mathbb
\usepackage[margin=1in]{geometry} % Set all margins to 1 inch

% Custom paragraph formatting
\setlength{\parskip}{1em} % Add space between paragraphs
\setlength{\parindent}{0pt} % Remove paragraph indentation


\begin{document}

\section*{Supplementary text: Rescorla-Wagner model on discrimination task}

We assume that the discrimination task reflects value learning, specifically how individuals update their weights (\(w\)) for each stimulus pair across trials. Learning is assumed to begin at 0.5 (indicating random knowledge of which stimulus in a pair leads to a reward) and progress toward 1 (indicating perfect knowledge). Each Day 1 session consists of five runs, with 24 trials per run. There are 12 different stimulus pairs, with each pair presented twice per run, counterbalanced for left/right positions. 

We specified and estimated three hierarchical learning models:

\begin{enumerate}
    \item by-Sess model has learning rates varied by session number (1,2,3)

\begin{itemize}
    \item Learning rates (\(\alpha_{j,c,p}\)) vary by stimulus pair but are constrained by a higher-level parameter (\(a_{j,c}\)) specific to each subject (\(j\)) and session (\(c\)), such that
    
\[
\alpha_{j,c,p} \sim \text{Beta}\left(a_{j,c} \cdot k_j, \, (1 - a_{j,c}) \cdot k_j\right),
\]
    where:
    \begin{itemize}
        \item \(j\): Subject,
        \item \(c\): Session (1, 2, 3),
        \item \(p\): Cue pair (1 to 12).
    \end{itemize}

\item 
The higher-level learning rates (\(a_{j,c}\)) are assumed to follow:
\[
a_{j,c} \sim \text{Beta}\left(\mu_c \cdot \kappa, \, (1 - \mu_c) \cdot \kappa\right), \quad 
\]

\[
k_j \sim \text{Gamma}(1, 0.1), \quad \text{for each } j = 1, \dots, \text{nsubs},
\]

where:
\begin{itemize}
    \item \(\mu_c\) is the session-specific mean with its prior drawn from a Beta distribution,
    \[
\mu_c \sim \text{Beta}(8, 2), \quad \text{for each } c = 1, 2, 3.
\]

    \item \(\kappa\) is a global scaling parameter that determines the precision of the distribution.

    \[
\kappa \sim \text{Gamma}(1, 0.1),
\]

\end{itemize}

\end{itemize}


For any stimulus pair, if the current trial (\(i\)) involves this pair, the associated weight (\(w\)) is updated on a trial-by-trial basis as follows:

\[
w_{i+1} = w_{i} + \alpha \cdot (1 - w_{i}),
\]

where \(\alpha\) represents the learning rate for the stimulus pair on the current trial. For brevity, subject, session, and stimulus pair indices for \(\alpha\) are omitted. The discrimination response  (\(\text{Resp}_{i}\); 1 for correctly choosing the odor-predictive stimulus; 0 otherwise) is modeled as:

\[
\text{Resp}_{i} \sim \text{Bernoulli}(w_{i}).
\]



       
\item by-TMS-condition model where learning rates are varied by day 1 TMS condition (sham, cTBS)

    
\item Same model: constant learning rates for each subj?


\end{enumerate}


\end{document}
